\section{Mathematical Formulation}
We adopt the classical SIR (Susceptible–Infected–Recovered) model to simulate the dynamic evolution of states within a spatial grid. Each grid cell maintains the proportion of its population in the three states:

\begin{itemize}
    \item \textbf{S}: Susceptible individuals
    \item \textbf{I}: Infected individuals
    \item \textbf{R}: Recovered individuals
\end{itemize}

The local dynamics of state transitions are governed by the standard system of differential equations:

\begin{align*}
\frac{dS}{dt} &= -\beta S \cdot I \\
\frac{dI}{dt} &= \beta S \cdot I - \gamma I \\
\frac{dR}{dt} &= \gamma \cdot I
\end{align*}

To model spatial interaction across neighboring cells, we extend the equations by incorporating a coupling term that accounts for the influence of infected populations in adjacent locations. The modified equation for the susceptible population in cell \((i, j)\) becomes:

\[
\frac{dS_{ij}}{dt} = -\beta S_{ij} \cdot \left( I_{ij} + \sum_{(k,l) \in N(i,j)} I_{kl} \right)
\]

where \( N(i,j) \) denotes the set of neighboring cells of \((i,j)\). Similar formulations are used for \( \frac{dI_{ij}}{dt} \) and \( \frac{dR_{ij}}{dt} \) to capture spatially distributed SIR dynamics.
